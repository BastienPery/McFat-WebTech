\documentclass[12pt]{article}


\usepackage[utf8x]{inputenc}
\usepackage{graphicx}
\usepackage[english]{babel}

\usepackage{geometry}
\geometry{hmargin=2.5cm,vmargin=2.5cm}

\usepackage{comment}

\usepackage{xcolor}
\definecolor{color_section}{RGB}{230,8,31}
\definecolor{color_subsection}{RGB}{252,170,50}

\usepackage{hyperref}

\begin{document}
    \begin{titlepage}
        \title{Web technology project plan}
        \author{Mac group}

        \begin{figure}
            \begin{minipage}[c]{0.40\linewidth}
                \includegraphics[width=0.7\textwidth]{TUGraz.png}
            \end{minipage}
            \hfill
            \begin{minipage}[c]{0.48\linewidth}
                \hfill
                \includegraphics[width=0.7\textwidth]{Erasmus.jpeg}
            \end{minipage}
        \end{figure}

        \begin{center}
            \includegraphics[width=0.5\textwidth]{mcdonalds-fat-logo.jpg}\\
            \vspace{1cm}
            \Large McFat Web Application Documentation\\
            \rule{5cm}{.5pt}\\
            \vspace{0.6cm}
            {\Large Web technology}\\
            \vspace{0.1cm}
            {\Large Winter semester 2018-2019} \\
            \vspace{5em}

            \textbf{Jens Mellberg - Matteo Milo$\mathbf{\check{s}}$ - Bastien Pery}\\
        \end{center}

    \end{titlepage}

    \renewcommand{\contentsname}{\color{color_section}\\Table of contents}
    \tableofcontents

    \newpage

    {\color{color_section}\section{Application description}}
    Mac group was composed by four students and four different nationalities.
        Croatian, French, Swedish and Finnish.
        Unfortunately, one of us drop the course just in the middle. 
        But, As part of our Erasmus exchange program, 
        we wanted to develop a web application which could be used by a lot 
        of people around the world and especially by Erasmus people.\\

        \noindent Our application reached this goal because, thanks to it,
        every european student is now able to find important 
        information about McDonald in Europe.\\

    \noindent Our McFat Web Application is divided into four different parts, each including different functionalities:
    \begin{itemize}
        \item MacHome
        \item MacMap
        \item MacStats
        \item MacMenu
    \end{itemize} 

    {\color{color_section}\section{Functionalities}}

        {\color{color_subsection}\subsection{MacHome}}
        \noindent The MacHome page is basically a simple home page which includes 3
        different things. A McDonald logo, a brief description of the
        application and a navigation bar which helps you to reah any other
        part of our application.\\

        {\color{color_subsection}\subsection{MacMap}}
        \noindent The MacMap page is a little bit more complex. Indeed, like
        all other pages, it includes the navigation bar but also a map and one
        button.\\

        \noindent The map is rendered by an external map API provided by 
        \url{https://www.esri.com}. Thanks to this API, you can easily
        navigate on the map, zoom-in or zoom-out on it. Also, you can see 
        that almost all european McDonald locations are displayed on it.
        (We linked some data with the API). If you click on one location,
        it will display the latitude, the longitude and the city of this
        location.\\

        \noindent The second thing on this page is the button "Display your
        coordinates and find the nearest McDonald!". If you click on this button
        our application will give you your location (if you accept it) and give
        you how far is the nearest McDonald and in which city!

        \newpage

        {\color{color_subsection}\subsection{MacStats}}
        \noindent On the MacStats page you can find some new different elements.
        Indeed, on this page, thanks to 2 different forms, users can display
        some statistics about European McDonalds.\\

        \noindent For example, you can select as many countries as you wish
        and choose which data you want to know such as number of restaurants
        in these countries, number of restaurants per 100k inhabitants... When
        you click on the button "Draw  Statistic", some charts will appear and
        show you the data you asked for.\\
        {\color{color_subsection}\subsection{MacMenu}}
        \noindent The MacMenu page is the page where user can create and see his order.
         Meals are divided into 5 categories: main meals, side meals, salads, drinks 
         and deserts. Each user can add as much meals of each category as he wants. 
         Meals are added using drag-and-drop functionality. As user is adding his meals,
          concurrently our web application is calculating how much calories will users 
          order have. At anytime, user can remove any of his meals from the order by 
          just clicking on the picture of the meal. As the meal is removed, also the 
          sum of calories will be downgraded for appropriate amount. When user is 
          finished with his order, if he wants, he can write his name and email, and 
          order will be sent to his email address.\\

    {\color{color_section}\section{Web technologies used}}

    \noindent In our application we used a lot of different web technologies.\\

    \noindent First of all AJAX. Our application is an asynchronous web application.
    To perform this task, we used the jQuery API.\\

    \noindent We also used some HTML5 technologies which were required:\\
    \begin{itemize}
        \item Canvas as graphic technology for the statistic charts on MacStats
        \item WebStorage as database to store all the McDonald locations and do some analysis on it
        \item WebWorkers on the MacMenu page which calculate how long we need to create an order
        \item The Geolocation API to find the user's location on MacMap
        \item Some HTML5 forms on MacStats
        \item The drag and drop HTML5 techonlogy on MacMenu.
    \end{itemize}


    {\color{color_section}\section{Separation of concerns}}

    \noindent Separation of Concerns(SoC) refers to the idea that computer programs should be separated into distinct sections so that each section addresses a specific concern. By distinguishing parts of our code within well-defined interfaces, we can create programs that have distinct subroutines mapped out into more readable terms. In doing so, we can maintain order within our program while allowing for more concise logical conclusions on a step-by-step basis. \\
   
    \noindent In our web application, we used Web Storage as model part of our app. We were using local storage functionality, which stores data with no expiration date. In our database we were saving two types of objects, one which had latitude, longitude and city of each McDonald's restaurant, and one which was containing type, name, and amount of calories for each of our meal. That data was accessed from controllers we implemented in JavaScript. After we have loaded that data, we were using a lot of different computations based on it in our controllers. All that was necessary because of view part of our app, which we implemented as html files. There we have rendered all the parts of our app so it is visible to users. Also, there were some dynamic computations we needed based on user activity, such as calculation of total calories, so after every meal that was added to the order, our view and controller part communicated because of calculating and rendering new sum of calories.\\

    {\color{color_subsection}\subsection{MVC}}

    \noindent As we said before, we used a lot of different technologies with different
    roles in our application. To make it maintainable and easier to understand
    for developers, we divided our all application following the MVC pattern.\\

    \noindent We have different controllers which are controlling views update,
    making computations and be the link between the view and the database.\\

\end{document}