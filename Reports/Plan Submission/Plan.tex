\documentclass[12pt]{article}


\usepackage[utf8x]{inputenc}
\usepackage{graphicx}
\usepackage[english]{babel}

\usepackage{geometry}
\geometry{hmargin=2.5cm,vmargin=2.5cm}

\usepackage{comment}

\usepackage{xcolor}
\definecolor{color_section}{RGB}{230,8,31}
\definecolor{color_subsection}{RGB}{252,170,50}

\begin{document}
    \begin{titlepage}
        \title{Web technology project plan}
        \author{Mac group}

        \begin{figure}
            \begin{minipage}[c]{0.40\linewidth}
                \includegraphics[width=0.7\textwidth]{./Pictures/TUGraz.png}
            \end{minipage}
            \hfill
            \begin{minipage}[c]{0.48\linewidth}
                \hfill
                \includegraphics[width=0.7\textwidth]{./Pictures/Erasmus.jpeg}
            \end{minipage}
        \end{figure}

        \begin{center}
            \includegraphics[width=0.5\textwidth]{./Pictures/mcdonalds-fat-logo.jpg}\\
            \vspace{1cm}
            \Large McFat Web Application Plan\\
            \rule{5cm}{.5pt}\\
            \vspace{0.6cm}
            {\Large Web technology}\\
            \vspace{0.1cm}
            {\Large Winter semester 2018-2019} \\
            \vspace{3em}

            \textbf{Jens Mellberg - Matteo Milo$\mathbf{\check{s}}$}\\
            \textbf{Bastien Pery - Henrikki Helander}
        \end{center}

    \end{titlepage}

    \renewcommand{\contentsname}{\color{color_section}\\Table of contents}
    \tableofcontents

    \newpage

    {\color{color_section}\section{Group presentation}}
        Mac group is composed by four students and four different nationalities.
        Swedish, Croatian, Finnish and French. As part of our Erasmus exchange program, 
        we want to develop a web application which could be used by a lot 
        of people around the world and especially by Erasmus people.
        
        \begin{figure} [h!]
            \begin{minipage}[c]{0.28\linewidth}
                \includegraphics[width=0.7\textwidth]{./Pictures/cro.png}
            \end{minipage}
            \hfill
            \begin{minipage}[c]{0.22\linewidth}
                \includegraphics[width=0.7\textwidth]{./Pictures/france.png}
            \end{minipage}
            \hfill
            \begin{minipage}[c]{0.23\linewidth}
                \hfill
                \includegraphics[width=0.7\textwidth]{./Pictures/sweden.png}
            \end{minipage}
            \hfill
            \begin{minipage}[c]{0.24\linewidth}
                \hfill
                \includegraphics[width=0.7\textwidth]{./Pictures/finnish.png}
            \end{minipage}
        \end{figure}

        \vspace{-1em}

    {\color{color_section}\section{Plan presentation}}
        {\color{color_subsection}\subsection{Web app description}}
        
        Usually, when (young) people go to some foreign country, and they
        are not sure where and what to eat, the most simple choice is to
        choose a McDonalds restaurant. Therefore, we decided to create a web 
        app that will help us finding the nearest McDonalds restaurant, 
        but that will also give us some other important information.\\
        
        \noindent For example, we might be interested in whether or not the McDonalds we want 
        to visit has a drive in option, or maybe we just want to find out 
        the exact calorie content of meals we want to eat. Also, additional 
        information that we might find useful would be comparing prices from 
        different countries, so we will know how much more or less money we would be 
        paying compared to our home country. If we feel that we 
        have the time, we will add more functionality along the way.\\

        \noindent So our first feature will be to implement a map that will provide
        the nearest MacDonald restaurant to the user thanks to his current coordonates.
        (We already found open source databases for all restaurants in Europe and USA.)\\

        \noindent Our second feature will be to compare the prices of MacDonalds meals
        in each country and show them by drawing a dynamic graphic. We will be
        able to select countries we want to compare. If we can find all the
        required databases we will also be able to compare the differences
        between availables meals in each restaurant.\\

        \noindent Our third feature will be a menu picker that will help you to choose
        your menu depending on what you want (cheapest, fattest... whatever).
        And show the price of it depending on the country (if we can find the
        appropriate database for that.)\\

        \noindent That is what our application's core will compose of. As we said, if we feel
        that we have the time to implement more features, we will.

        \newpage

        {\color{color_subsection}\subsection{Planning and Roles}}
        \vspace{1em}
           
	\begin{center}
	    \begin{tabular}{ |c|c|c|c| } 
 	        \hline
	        \textbf{Task} & \textbf{Estimated deadline} & \textbf{Who} & \textbf{Estimated Time}  \\
	        \hline
 	        Project plan & until 18.11. & Bastien and Matteo & 3 hours \\ 
 	        \hline
 	        Plan presentation & until 18.11. & Henrikki and Jens & 2 hours \\
            \hline 
            Result presentation & until 30.01. & everyone & 5 hours \\ 
            \hline 
        \end{tabular}
    \end{center}
    
    \begin{center}
        \begin{tabular}{ |c|c|c|c| } 
            \hline
            \textbf{Implementation} & \textbf{Estimated deadline} & \textbf{Who} & \textbf{$\approx$ 80 hours} \\
            \hline
            Responsive map & 14.12 & 2 people & 10 hours each \\
            \hline
 	        Statistical drawings & 10.01 & 2 people & 10 hours each\\  
            \hline
            Menu picker & 20.01 & 2 people & 10 hours each \\
            \hline
            Other features & 30.01 & 2 people & 10 hours each \\
	        \hline
	    \end{tabular}
    \end{center}
    
    \begin{center}
        \begin{tabular}{ |c|c|c|c| } 
            \hline
            \textbf{Documentation} & \textbf{Estimated deadline} & \textbf{Who} & \textbf{$\approx$ 12 hours} \\
            \hline
            First drafts & 01.12 & Bastien & 3 hours \\
            \hline
 	        Technologies description & 30.01 & 2 people & 3 hours\\  
            \hline
            User manual & 30.01 & 2 people & 4 hours \\
            \hline
            Other documents & 30.01 & 2 people & 2 hours \\
	        \hline
	    \end{tabular}
    \end{center}
    
    {\color{color_subsection}\subsection{Member responsabilities}}

    \noindent As long as we will have new features to implement, we will work by teams
    of two people on each. In that way, we will be more efficient and more
    focus in our task. We have not decided who is going to do which task yet.
    We will decide it as soon as possible.\\

    \noindent For our Mac Fat web application Bastien will certainly be the manager of
    the group. Making sure that every deadline is met and also keeping an eye 
    on all the work to be sure that we reach the goal.\\

\end{document}